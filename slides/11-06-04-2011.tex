\documentclass[norsk]{beamer}
\usepackage[latin1]{inputenc}

\usepackage{listings}


\lstset{language=Java}

%\usetheme{Warsaw}
\usetheme[sky-200]{UiO}

%\useinnertheme[shadow]{rounded}
\urlstyle{sf}

\title[INF1010]%(optional, only for long titles)
{INF1010 - Felles�velser}
\subtitle{Praktisk gjennomgang av programmeringsteknikker}
\author[Tor Ivar Johansen] % (optional, for multiple authors)
{Tor Ivar Johansen}
\institute[Universitetet i Oslo] % (optional)
{
Institutt for informatikk
}
\subject{Objektorientert programmering}

% \AtBeginSection[]
% {
%    \begin{frame}
%        \frametitle{Disposisjon}
%        \tableofcontents[currentsection]
%    \end{frame}
% }



\begin{document}

\begin{frame}
\huge INF1010 - Felles�velse
\end{frame}

 \begin{frame}
 \titlepage 
 \end{frame}

\begin{frame}
\frametitle{Rekursjon cont.}
\begin{itemize}
\item Sudoku
\item Tree Sort
\item Dronningrekursjon
\end{itemize}
\end{frame}

\begin{frame}
  \frametitle{F�rst litt generell info}
\end{frame}

\begin{frame}
  \textbf{Rekursjon} \\
       \hspace{2cm}se rekursjon
\end{frame}

\begin{frame}
  \textbf{Rekursjon} \\
       \hspace{2cm}om du fortsatt ikke forst�r: se rekursjon
\end{frame}


\begin{frame}
\frametitle{Tree Sort}
  \begin{itemize}
  \item Perfekt for sortering.
  \item Den faktiske sorteringen foreg�r under innsetting.
  \item Vi bare henter ut i rekkef�lgen vi �nsker.
  \end{itemize}
\end{frame}

% \begin{frame}
%   \textbf{Rekursjon} \\
%        \hspace{2cm}se rekursjon
% \end{frame}

% \begin{frame}
%   \textbf{Rekursjon} \\
%        \hspace{2cm}om du fortsatt ikke forst�r: se rekursjon
% \end{frame}

%http://www.post-literate.com/gerpunx/archives/2005/01/prepare_to_lose_your_mind.php


\end{document}
