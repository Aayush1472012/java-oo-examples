\documentclass[norsk]{beamer}
\usepackage[latin1]{inputenc}

\usepackage{listings}


\lstset{language=Java}

%\usetheme{Warsaw}
\usetheme[sky-200]{UiO}

%\useinnertheme[shadow]{rounded}
\urlstyle{sf}

\title[INF1010]%(optional, only for long titles)
{INF1010 - Felles�velser}
\subtitle{Praktisk gjennomgang av programmeringsteknikker}
\author[Tor Ivar Johansen] % (optional, for multiple authors)
{Tor Ivar Johansen  \and Karoline Lunder \and Magnus D�hlen}
\institute[Universitetet i Oslo] % (optional)
{
Institutt for informatikk
}
\subject{Objektorientert programmering}

% \AtBeginSection[]
% {
%    \begin{frame}
%        \frametitle{Disposisjon}
%        \tableofcontents[currentsection]
%    \end{frame}
% }



\begin{document}

\begin{frame}
\huge INF1010 - Felles�velse
\end{frame}

 \begin{frame}
 \titlepage 
 \end{frame}

\begin{frame}
\frametitle{Tr�der}
\begin{itemize}
\item Hva
\item Hvordan
\end{itemize}
\end{frame}

\begin{frame}
  \frametitle{Tr�der}
  \begin{itemize}
  \item Tr�der er lettvekt prosesser
  \item Tr�der h�ndteres av OS-kjernen
  \item Tr�der deler minne og ressurser
  \item Tr�der kan switches betydelig raskere
  \end{itemize}
\end{frame}

\begin{frame}
  \frametitle{Tr�der i java}
  \begin{itemize}
  \item Du har alltid en tr�d, main tr�den
  \item Derfra kan du lage flere tr�der.
  \item Du kan ogs� prioritere tr�der over andre.
  \end{itemize}
\end{frame}

\begin{frame}
  \frametitle{Tr�der i java}
  \begin{itemize}
  \item Kan lages p� to m�ter
    \begin{itemize}
    \item extend \textbf{Thread} klassen
    \item implement \textbf{Runnable} grensesnittet
    \end{itemize}
  \end{itemize}
\end{frame}

\begin{frame}
  \frametitle{Extend \textbf{Thread} klassen}
  \begin{itemize}
  \item M� override \texttt{run()} metoden.
  \item Objektet instansieres og \texttt{start()} metoden m� kalles
  \item Kj�ringen startes s� av Java Runtime
  \end{itemize}
\end{frame}

\begin{frame}
  \frametitle{implement \textbf{Runnable} grensesnittet}
  \begin{itemize}
  \item M� implementere \texttt{run()} metoden.
  \item M� instansiere nytt \textbf{Thread} objekt og sende seg selv med som parameter
  \item \texttt{start()} metoden m� kalles for videre h�ndtering av Java Runtime
  \end{itemize}
\end{frame}

\begin{frame}
  \frametitle{Hva er best?}
  \begin{itemize}
  \item Opptil deg selv.
  \item extend-alternativet gir mindre kode, men hindrer deg i � extende andre klasser
  \end{itemize}
\end{frame}

\end{document}
